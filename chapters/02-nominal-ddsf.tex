\chapter{Nominal Data-Driven Safety Filter}\label{chap:nominal-ddsf}

In this chapter, we focus on the design of safety filters for noise-free LTI system.
We will first introduce the formulation of model-based safety filter and nominal data-driven safety filter, respectively.
Then we show that the nominal data-driven safety filter is equivalent to the model-based safety filter for noise-free LTI system.
Finally, we discuss the possible improvements on the designing of nominal data-driven safety filter.


\section{Model-Based and Data-Driven Safety Filter for LTI system}\label{sec:formulation-nominal}

As discussed in \cref{chap:introduction}, the predictive safety filter \cref{eq:state-based-safety-filter} is designed based on the assumption that the system state is fully observable.
Since we are working with noise-free LTI system where only the output is observable, we need to reformulate the safety filter \cref{eq:state-based-safety-filter} to an output-based safety filter as follows:

\begin{subequations}
\label{eq:model-based-safety-filter}
\begin{align}
    \min_{\substack{\bar{u}, \bar{y} \\ \bar{x}}} \quad & \norm{\baru_{0} - u^{\text{obj}}}_R^2 \label{eq:model-based-safety-filter-cost} \\
    \textrm{s.t.} \quad & 
    \bar{x}_{k+1} = A \bar{x}_k + B \bar{u}_k \nonumber \\
    &
    \bar{y}_k = C \bar{x}_k + D \bar{u}_k,  \quad k \in \left[-n, L-1\right] \label{eq:model-based-safety-filter-dynamics} \\
    & 
    \begin{bmatrix}
        \subseq{\bar{u}}{-n}{-1} \\
        \subseq{\bar{y}}{-n}{-1} \\
    \end{bmatrix} = 
    \begin{bmatrix}
        \subseq{u}{t-n}{t-1} \\
        \subseq{y}{t-n}{t-1} \\
    \end{bmatrix} \label{eq:model-based-safety-filter-initial}\\
    & 
    \begin{bmatrix}
        \subseq{\bar{u}}{L-n}{L-1} \\
        \subseq{\bar{y}}{L-n}{L-1} \\
    \end{bmatrix} = \begin{bmatrix}
        \vRepeatVec{\us}{n} \\
        \vRepeatVec{\ys}{n} \\
    \end{bmatrix} \label{eq:model-based-safety-filter-terminal}\\
    &
    \bar{u}_k \in \Uset, \quad k \in \left[0, L-1\right] \label{eq:model-based-safety-filter-input}\\
    &
    \bar{y}_k \in \Yset, \quad k \in \left[0, L-1\right] \label{eq:model-based-safety-filter-output}
\end{align}
\end{subequations}

Where $\us$ and $\ys$ are a pair of safe input-output equilibrium, as defined in \cref{def:input-output-equilibrium} and \cref{remark:safe-equilibrium}.
It still follows the predictive safety filter idea: evaluate the safety of a proposed input sequence by predicting future trajectory of the system, and the objective \cref{eq:model-based-safety-filter-cost} is also designed to minimize the deviation from the given control input.
But there are some further discussions needed for this formulation:

\begin{enumerate}
    \item The optimization variables not only include candidate input sequence $\baru$ and output sequence $\bary$, but also the state sequence $\barx$.
    This is necessary to formulate the system dynamics.
    \item Since we only observe the output of the system, more than one step of input-output pairs are needed to uniquely determine the state of the system.
    This is reflected in \cref{eq:model-based-safety-filter-initial}.
    Here we choose to use the most recent $n$ steps of input-output pairs to uniquely determine the inner state of the system.
    \item The terminal constraint \cref{eq:model-based-safety-filter-terminal} is less general than \cref{eq:state-based-safety-filter-terminal}.
    As discussed in \cref{remark:safe-traj}, a long enough safe trajectory can be used to ensure the safety of system in all future time steps.
    Here we use a safe input-output equilibrium to construct such a safe trajectory.
    More complicated terminal constraints can be used, as will be briefly discussed in \cref{sec:improvements-nominal}.
    \item Due to the form of terminal condition \cref{eq:model-based-safety-filter-terminal}, the prediction horizon $L$ needs to be larger than $n$.
    In fact, since the last $n$ steps of input-output pairs are used to determine the terminal state $\xs$ at time step $L-n$, the actual prediction horizon (time steps before the system is constraint to be inside the terminal set) is $L-n$.
\end{enumerate}

This formulation also provides closed-loop guarantees similar to the state-based safety filter \cref{eq:state-based-safety-filter}.

\begin{theorem}[Closed-loop Guarantees for Model-based Safety Filter]\label{thm:guarantee-model-based-lti}
    When the system is equipped with the safety filter \cref{eq:model-based-safety-filter}, and a one-step receding horizon scheme, closed-loop constraint satisfaction and recursive feasibility hold.
\end{theorem}

\begin{proof}[Proof of \Cref{thm:guarantee-model-based-lti}]\prooflabel[theorem]{thm:guarantee-model-based-lti}{prf:guarantee-model-based-lti}
    First we show constraint satisfaction for the next time step, then we show recursive feasibility by providing a feasible solution for the next time step.
    Then the closed-loop constraint satisfaction for all future time steps is shown by induction.

    Due to initial constraint \cref{eq:model-based-safety-filter-initial} and dynamic constraint \cref{eq:model-based-safety-filter-dynamics}, the initial state of the system $x_t$ is the same as the optimal solution $\starx_0(t)$.
    Then due to dynamic constraint \cref{eq:model-based-safety-filter-dynamics}, after $\staru_0$ is applied to the real system, the output will be $\stary_0(t)$, which satisfies the constraint $\Yset$ due to \cref{eq:model-based-safety-filter-output}.

    At time step $t+1$, we define the candidate solution $\hatu(t+1)$, $\hatx(t+1)$, and $\haty(t+1)$ as follows:
    {
    \setlength{\abovedisplayskip}{3pt}
    \setlength{\belowdisplayskip}{3pt}
    % \setlength{\abovedisplayshortskip}{0pt}
    % \setlength{\belowdisplayshortskip}{0pt}
    \begin{align*}
        \subseq{\hatu}{-n}{L-2}(t+1) = \subseq{\staru}{-n+1}{L-1}(t), \quad \hatu_{L-1}(t+1) = \us \\
        \subseq{\hatx}{-n}{L-2}(t+1) = \subseq{\starx}{-n+1}{L-1}(t), \quad \hatx_{L-1}(t+1) = \xs \\
        \subseq{\haty}{-n}{L-2}(t+1) = \subseq{\stary}{-n+1}{L-1}(t), \quad \haty_{L-1}(t+1) = \ys
    \end{align*}
    }
    where $\xs$ is the equilibrium state implicitly defined by $\us$ and $\ys$.
    Due to terminal constraint \cref{eq:model-based-safety-filter-terminal}, we have $\hatx_{L-2}(t+1) = \xs$, also because the optimal solution $\staru(t)$, $\starx(t)$, and $\stary(t)$ satisfy the dynamic constraint \cref{eq:model-based-safety-filter-dynamics}, we can conclude that $\subseq{\hatu}{-n}{L-1}(t+1)$, $\subseq{\hatx}{-n}{L-1}(+1)$, and $\subseq{\haty}{-n}{L-1}$ is a trajectory of the system, thus satisfies the dynamic constraint \cref{eq:model-based-safety-filter-dynamics}.

    \Crefrange{eq:model-based-safety-filter-initial}{eq:model-based-safety-filter-terminal} are satisfied by the construction of the solution.
    \Crefrange{eq:model-based-safety-filter-input}{eq:model-based-safety-filter-output} are also satisfied by the candidate solution, since they are satisfied by the optimal solution at time step $t$, and we have $\us \in \Uset$, $\ys \in \Yset$.

    So recursive feasibility is shown.
\end{proof}

\begin{remark}\label{remark:usage-output-safety-filter}
    As seen in the formulation \cref{eq:model-based-safety-filter}, the safety filter needs more than one step of input-output pairs as the initial condition.
    This can be problematic at first time steps, when there are not enough input-output pairs available.
    To overcome this problem, we can either run the system without a safety filter for a few time steps, or use expert knowledge to construct a proper initial condition.
\end{remark}

Based \cref{lemma:fundamental-lemma}, the dataset trajectory can be used to reconstruct the system dynamic constraint in the model-based safety filter \cref{eq:model-based-safety-filter}.
So we can formulate the nominal data-driven safety filter as:

\begin{subequations}
\label{eq:nominal-ddsf} 
\begin{align}
    \min_{\alpha, \bar{u}, \bar{y}} \quad & \norm{\baru_{0} - u^{\text{obj}}}_R^2  \label{eq:nominal-ddsf-cost}\\
    \textrm{s.t.} \quad & 
    \begin{bmatrix}
        \subseq{\bar{u}}{-n}{L-1} \\
        \subseq{\bar{y}}{-n}{L-1} \\
    \end{bmatrix} = 
    \begin{bmatrix}
        \hankel{L+n}{u^d} \\
        \hankel{L+n}{y^d} \\
    \end{bmatrix} \alpha \label{eq:nominal-ddsf-dynamics}\\
    & 
    \begin{bmatrix}
        \subseq{\bar{u}}{-n}{-1} \\
        \subseq{\bar{y}}{-n}{-1} \\
    \end{bmatrix} = 
    \begin{bmatrix}
        \subseq{u}{t-n}{t-1} \\
        \subseq{y}{t-n}{t-1} \\
    \end{bmatrix} \label{eq:nominal-ddsf-initial}\\
    & 
    \begin{bmatrix}
        \subseq{\bar{u}}{L-n}{L-1} \\
        \subseq{\bar{y}}{L-n}{L-1} \\
    \end{bmatrix} = 
    \begin{bmatrix}
        u_n^s \\
        y_n^s \\
    \end{bmatrix} \label{eq:nominal-ddsf-terminal}\\
    &
    \bar{u}_k \in \Uset, \quad k \in \left[0, L-1\right] \label{eq:nominal-ddsf-input}\\
    &
    \bar{y}_k \in \Yset, \quad k \in \left[0, L-1\right] \label{eq:nominal-ddsf-output}
\end{align}
\end{subequations}

As described above, the constraint \cref{eq:nominal-ddsf-dynamics} replaces the system dynamic constraint \cref{eq:model-based-safety-filter-dynamics} in the model-based safety filter.
Where $u^d$ and $y^d$ represent a dataset trajectory of length $N$: $\datasetSequence{u}{N}$ and $\datasetSequence{y}{N}$.

And we have the equivalence between \cref{eq:model-based-safety-filter} and \cref{eq:nominal-ddsf} for noise-free LTI system.

\begin{theorem}[Equivalence of Data-driven and Model-based Safety Filters]\label{thm:nominal-equivalence}
    Suppose:
    \begin{enumerate}
        \item The system is a controllable LTI system.
        \item The dataset input sequence $\datasetSequence{u}{N}$ is persistently exciting with order $(L+n)+n$, where $L$ is the prediction horizon and $n$ is order of the system.
    \end{enumerate}
    Then we have: any feasible solution of \cref{eq:nominal-ddsf} can be transformed to a feasible solution of \cref{eq:model-based-safety-filter} with the same candidate input sequence, and vice versa.
\end{theorem}

\begin{proof}[Proof of \Cref{thm:nominal-equivalence}]\prooflabel[theorem]{thm:nominal-equivalence}{prf:nominal-equivalence}
    Since the constraints \crefrange{eq:model-based-safety-filter-initial}{eq:model-based-safety-filter-output} and \crefrange{eq:nominal-ddsf-initial}{eq:nominal-ddsf-output} are identical, we only need to show that the dynamic constraints \cref{eq:model-based-safety-filter-dynamics} and \cref{eq:nominal-ddsf-dynamics} can satisfied with the same candidate input sequence.

    First suppose $\baru_D$ and $\bary_D$ is a feasible solution of \cref{eq:nominal-ddsf}.
    Then by \cref{lemma:fundamental-lemma} and the conditions of \cref{thm:nominal-equivalence}, we can conclude that $\baru_D$ and $\bary_D$ form a trajectory of the system, of length $L+n$.
    Then we can define $\barx_D$ as the state sequence of the system corresponding to $\baru_D$ and $\bary_D$.
    Since $\baru_D$, $\barx_D$ and $\bary_D$ is a trajectory of the system, it satisfies the dynamic constraint \cref{eq:model-based-safety-filter-dynamics}.

    Then suppose $\baru_M$, $\barx_M$ and $\bary_M$ is a feasible solution of \cref{eq:model-based-safety-filter}.
    Then due to the dynamic constraint \cref{eq:model-based-safety-filter-dynamics}, $\baru_M$, $\barx_M$ and $\bary_M$ is a trajectory of the system, of length $L+n$.
    Again due to \cref{lemma:fundamental-lemma} and the conditions of \cref{thm:nominal-equivalence}, we can conclude that $\baru_M$ and $\bary_M$ is a feasible solution of \cref{eq:nominal-ddsf}, with proper choice of $\alpha$.
\end{proof}

Combining \cref{thm:guarantee-model-based-lti} and \cref{thm:nominal-equivalence}, we can conclude that the nominal data-driven safety filter \cref{eq:nominal-ddsf} also provides closed-loop guarantees for controllable LTI system.


\section{Improvements on Nominal Data-Driven Safety Filter}\label{sec:improvements-nominal}

For the nominal data-driven safety filter \cref{eq:nominal-ddsf}, the most important conservative comes from the terminal constraint \cref{eq:nominal-ddsf-terminal}.
Generally speaking, we only need the last $n$ steps of input-output pairs $\subseq{\baru}{L-n}{L-1}$ and $\subseq{\bary}{L-n}{L-1}$ to be a safe trajectory as defined in \cref{def:safe-traj}.
By the definition of safe trajectory, we can also use a proof similar to \cref{prf:guarantee-model-based-lti} to show that with this modification of terminal constraint, the safety filter \cref{eq:nominal-ddsf} still provides closed-loop guarantees for controllable LTI system.

However, it is generally difficult to find the set of all safe trajectories of length $n$ for a given system, even if the system is LTI and controllable.
It is easier and possible to find an \emph{Invariant Set} of the system instead.
As discussed in \cite{berberichDesignTerminalIngredients2021}, the invariant set can be defined with respect to the \emph{extended state} of the system, which contains $n$ input-output pairs of the system.
The main drawback of this method includes: first it is required that $l \times p = n$; second it only provides method of estimating the invariant set form \emph{noise-free dataset}.

Further discussion about the invariant set in an input-output manner, as well as the possibility of using noisy dataset to estimate the invariant set, is left for future work.

As will be discussed in \cref{chap:test-chronos}, it's also possible to use expert knowledge about a certain system to design specific terminal constraint, which can be used to improve the performance of the safety filter.
