% %---------------------------------------------------------------------------
% % Preface

% \chapter*{Preface}

% Blah blah \dots

%  \cleardoublepage

%---------------------------------------------------------------------------
% Table of contents

 \setcounter{tocdepth}{2}
 \tableofcontents

 \cleardoublepage

%---------------------------------------------------------------------------
% Abstract

\chapter*{Abstract}
 \addcontentsline{toc}{chapter}{Abstract}

Predictive safety filters are modular frameworks that can ensure safety of dynamic systems in terms of constraint satisfaction under arbitrary objective inputs.
This is achieved by proposing and applying safe inputs to the system, which are close to the objective inputs.
To ensure that the proposed inputs are safe, a model is usually required to predict the system's response.

In this thesis, we investigate the design, analysis, and application of different formulations of data-driven predictive safety filters.
They require no knowledge of system model, only some rich enough dataset trajectories are needed.

We show equivalence between the nominal data-driven safety filter and a model-based predictive safety filter for noise-free linear time invariant (LTI) systems, and proposed two different robust formulations for LTI systems with bounded noise.
Both formulations are shown to have qualitative guarantees, including closed-loop constraint satisfaction and recursive feasibility.
Their performance is illustrated with a numerical example.

We also propose a data-driven method to predict future trajectories of non-linear systems, and a metric to evaluate the richness of datasets based on their fractal dimensions.
Through a numerical example, we show that the proposed method yields more precise prediction compared to existing methods with rich enough dataset, and the metric is higher for richer dataset.

The data-driven safety filter is evaluated in the simulation of a miniature race car.
We propose different designs of terminal ingredient for the vehicle, and a practical scheme of dataset collection for general systems.

 \cleardoublepage

%---------------------------------------------------------------------------
% Symbols

% \chapter*{Nomenclature}\label{chap:symbole}
%  \addcontentsline{toc}{chapter}{Nomenclature}

% \section*{Symbols}
% \begin{tabbing}
%  \hspace*{1.6cm} \= \hspace*{8cm} \= \kill
%  $\mathrm{EHC}$ \> Conditional equation \> [$-$] \\[0.5ex]
%  $e$ \> Willans coefficient \> [$-$] \\[0.5ex]
%  $F,G$ \> Parts of the system equation \> [\unitfrac[]{K}{s}]
% \end{tabbing}

% \section*{Indicies}
% \begin{tabbing}
%  \hspace*{1.6cm}  \= \kill
%  a \> Ambient \\[0.5ex]
%  air \> Air
% \end{tabbing}

% \section*{Acronyms and Abbreviations}
% \begin{tabbing}
%  \hspace*{1.6cm}  \= \kill
%  NEDC \> New European Driving Cycle \\[0.5ex]
%  ETH \> Eidgen\"{o}ssische Technische Hochschule
% \end{tabbing}

%  \cleardoublepage

%---------------------------------------------------------------------------
