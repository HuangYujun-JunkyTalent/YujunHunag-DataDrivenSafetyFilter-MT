% %---------------------------------------------------------------------------
% % Preface

% \chapter*{Preface}

% Blah blah \dots

%  \cleardoublepage

%---------------------------------------------------------------------------
% Table of contents

 \setcounter{tocdepth}{2}
 \tableofcontents

 \cleardoublepage

%---------------------------------------------------------------------------
% Abstract

\chapter*{Abstract}
 \addcontentsline{toc}{chapter}{Abstract}

Predictive safety filters can keep the system safe under arbitrary given control input, by predicting future states of the system and applying a safe input that is close to the given input.
Generally, making predictions requires a model of the system, which requires much knowledge about the system and a complicated system identification process.
In this thesis, we investigate the design, analysis, and application of a novel data-driven predictive safety filter.
The proposed data-driven safety filter requires no knowledge of dynamics of the system, only some rich enough dataset trajectories are needed.

For noise-free Linear Time Invariant (LTI) system, the data-driven safety filter is equivalent to the model-based safety filter.
In this case, recursive feasibility and constraint satisfaction in close loop can be guaranteed following the same proof as the model-based safety filter.

The filter is also robust against bounded output noise.
Qualitative guarantees can be given, namely recursive feasibility and constraint satisfaction in close loop hold when bound of output noise is small enough.

For general non-linear system, the safety filter can also perform well in practice.
We propose a method of evaluating the quality of the dataset, as well as a method of improving the prediction performance by selecting and putting different weights on trajectory slices of the dataset.
These methods are tested on a simple non-linear system, it can be seen that with the proposed methods, better dataset yields better prediction performance.

The data-driven safety filter is also tested on the kinematic and dynamic model of Chronos vehicle.
We propose a scheme of dataset collection and test for the vehicle, which can be used to generate a rich enough dataset for the safety filter.
It can be seen that the safety filter can keep the vehicle within the given safe constraints, while keeping the actual input as close as possible to the desired input.

 \cleardoublepage

%---------------------------------------------------------------------------
% Symbols

% \chapter*{Nomenclature}\label{chap:symbole}
%  \addcontentsline{toc}{chapter}{Nomenclature}

% \section*{Symbols}
% \begin{tabbing}
%  \hspace*{1.6cm} \= \hspace*{8cm} \= \kill
%  $\mathrm{EHC}$ \> Conditional equation \> [$-$] \\[0.5ex]
%  $e$ \> Willans coefficient \> [$-$] \\[0.5ex]
%  $F,G$ \> Parts of the system equation \> [\unitfrac[]{K}{s}]
% \end{tabbing}

% \section*{Indicies}
% \begin{tabbing}
%  \hspace*{1.6cm}  \= \kill
%  a \> Ambient \\[0.5ex]
%  air \> Air
% \end{tabbing}

% \section*{Acronyms and Abbreviations}
% \begin{tabbing}
%  \hspace*{1.6cm}  \= \kill
%  NEDC \> New European Driving Cycle \\[0.5ex]
%  ETH \> Eidgen\"{o}ssische Technische Hochschule
% \end{tabbing}

%  \cleardoublepage

%---------------------------------------------------------------------------
