\chapter{Robust Data-Driven Safety Filter}\label{chap:robust-ddsf-lti}
TODO: Contains robust data-driven safety filter formulation and proof of qualitative guarantees.

In this chapter, we will discuss and compare different formulations of robust data-driven safety filters for LTI system with bounded output noise.
We will first introduce the direct formulation of the robust data-driven safety filter with rigorous constraint tightening scheme, and illustrate its limitations using a single-input-single-output (SISO) LTI system.
Then we will introduce the indirect formulation with qualitative constraint tightening scheme, and show that it can overcome the limitations of the direct formulation.

Before jumping into the details of data-driven safety filters, we first introduce the system we will be working with: LTI system with bounded output noise, as well as some notations that will be used later.

\begin{definition}[LTI System with Output Noise]\label{def:lit-output-noise}
    Starting from an LTI system without noise, as defined in \cref{def:dynamic-system} with linear dynamic and output functions, we can add a bounded output noise to the system.
    In another word, we can only observe $\tildey = y + \varepsilon$, where $\varepsilon$ is an unknown noise.

    We do not discuss or care about the distribution or stochastic properties of $\varepsilon$, the only condition we put on it is that it is bounded by a known constant $\barvareps$: $\infnorm{\varepsilon} \leq \barvareps$.
\end{definition}

\begin{remark}\label{remark:state-noise-lti}
    Although we do not explicitly deal with state noise in this chapter, we can incorporate bounded state noise when the system is \emph{asymptotically stable}.
    We do not discuss this point in detail in this Thesis.

    The intuition is, ``noise on the state'' will be transformed into ``noise on the output'' by the system dynamics.
    If the dynamics is asymptotically stable, this transformed noise on output will also be bounded.
\end{remark}

To simplify the discussion, we also assume the output constraint is a hypersquare, that is: $\Yset = \left\{y \, | \, \infnorm{y} \leq y_{\text{max}}\right\}$.

Within this chapter and the proof in appendix \cref{apps:prf-roubust-direct,apps:prf-roubust-indirect}, we will use the following notation:
Variables with tilde, like $\tildey$ represent observed variables with noise.
We will use $\varepsilon$ to represent noise variables, and $\barvareps$ to represent bound of the noise.
Note that usually we also extensively use variables with bar, like $\bary$, to represent the optimization variables.
This will not cause too much confusion, since noise variables are usually not optimization variables.

Also, similar to the notation in \cite{berberichDataDrivenRobust2021}, we use $\xi$ to denote the extended state of the system, which is a long vector that contains necessarily input-output pairs to determine the state of the system
For example, at time step $t$, the extended state can be written as $\xi_t = \transpose{\begin{bmatrix} \transpose{\subseq{u}{t-n}{t-1}} & \transpose{\subseq{y}{t-n}{t-1}} \end{bmatrix}}$.
Similar to other variables, it also has the version with output noise: $\tildexi_t = \transpose{\begin{bmatrix} \transpose{\subseq{u}{t-n}{t-1}} & \transpose{\subseq{\tildey}{t-n}{t-1}} \end{bmatrix}}$.
Note that we do not consider input noise in this Thesis, so the input is always exact.
Depending on the context, the number of input-output pairs used to construct $\xi$ can be $l$ or $n$.


\section{Direct Formulation}\label{sec:direct-formulation}

As has been discussed in \cite{coulsonDataenabledPredictiveControl2019}, when the dataset output sequence $\datasetSequence{\tildey}{N}$ is noisy, the Hankel matrix $\hankel{L+n}{\tildey^d}$ will become full rank, which means the column space of $\hankel{L+n}{\tildey^d}$ will be $\Real^{L+n}$.
In another word, the original data driven safety filter formulation \cref{eq:nominal-ddsf} will be able to make \emph{any prediction} possible.

To overcome this problem, it is advised in \cite{coulsonDataenabledPredictiveControl2019} to add a regularizer into the objective function.
As mentioned in \cref{sec:motivation-and-background}, we have a lot of discussions about choices and properties of different regularizers.
To allow more discussion about guarantees, we will use a \emph{2-norm} regularizer.

Also, as in common MPC schemes, it is also necessary to modify the constraint to deal with the noise.
We incorporate the constraint tightening scheme proposed and analysed in \cite{berberichRobustConstraintSatisfaction2020}.

The final point mention here is, we will consider the n-step receding horizon scheme, in contrast to the one-step receding horizon scheme used for nominal data-driven safety filter introduced in \cref{chap:nominal-ddsf}. 
Namely, the optimization problem \cref{eq:robust-ddsf-direct} will be solved every $n$ time steps, and the first $n$ inputs of the optimal solution will be applied to the system.

With all these modifications, the robust data-driven safety filter can be formulated as: 

\begin{subequations}
\label{eq:robust-ddsf-direct}
\begin{align}
    \min_{\substack{\alpha, \sigma \\ \baru, \bary}} \quad & \norm{\subseq{\baru}{0}{n-1} - \subseq{\uObj}{0}{n-1}(t)}_{\subseq{R}{0}{n-1}}^2 + \lamalpha \bar{\varepsilon} \norm{\alpha}_2^2 + \lamsigma \norm{\sigma}_2^2 \label{eq:robust-ddsf-direct-cost}\\
    \textrm{s.t.} \quad & 
    \begin{bmatrix}
        \subseq{\baru}{-n}{L-1} \\
        \subseq{\bary}{-n}{L-1} + \sigma \\
    \end{bmatrix} = 
    \begin{bmatrix}
        \hankel{L+n}{u^d} \\
        \hankel{L+n}{\tildey^d} \\
    \end{bmatrix} \alpha \label{eq:robust-ddsf-direct-dynamic}\\
    & 
    \begin{bmatrix}
        \subseq{\baru}{-n}{-1} \\
        \subseq{\bary}{-n}{-1} \\
    \end{bmatrix} = 
    \begin{bmatrix}
        \subseq{u}{t-n}{t-1} \\
        \subseq{\tildey}{t-n}{t-1} \\
    \end{bmatrix} \label{eq:robust-ddsf-direct-initial}\\
    & 
    \begin{bmatrix}
        \subseq{\baru}{L-n}{L-1} \\
        \subseq{\bary}{L-n}{L-1} \\
    \end{bmatrix} = 
    \begin{bmatrix}
        \vRepeatVec{\us}{n} \\
        \vRepeatVec{\ys}{n} \\
    \end{bmatrix} \label{eq:robust-ddsf-direct-terminal}\\
    &
    \baru_k \in \Uset, \quad k \in \left[0, L-1\right] \label{eq:robust-ddsf-direct-input}\\
    &
    \infnorm{\bary_k} + a_{1,k}\norm{\baru}_1 + a_{2,k}\norm{\alpha}_1 \nonumber \\
    &
    \quad + a_{3,k}\norm{\sigma}_1 + a_{4,k} \leq y_{\max}, \quad k \in \left[0, L-1\right] \label{eq:robust-ddsf-direct-output} \\
    &
    \norm{\sigma}_\infty \leq \bar{\varepsilon}\left( \norm{\alpha}_1 + 1 \right) \label{eq:robust-ddsf-direct-sigma}
\end{align}
\end{subequations}

where $\sigma$ is a slack variable introduced to deal with noise in dataset and online observation, $\lamalpha$ and $\lamsigma$ are two regularizing weights, $\subseq{\uObj}{0}{n-1}$ are $n$ objective inputs given by the arbitrary controller, $\subseq{R}{0}{n-1}$ represents the block-diagonal matrix with $n$ weighting matrices $R_0$ to $R_{n-1}$ on the diagonal, $\tildey^d$ is the noisy dataset output sequence, $a_{1,k}$ to $a_{4,k}$ are constraint tightening constants introduced in \cite{berberichRobustConstraintSatisfaction2020}, and $\bar{\varepsilon}$ is the bound of the noise.

We also need some discussion about the formulation:

\begin{enumerate}
    \item Since we are using the n-step receding horizon scheme, the first $n$ inputs of the optimal solution will be applied to the system.
    For the same reason, we need $n$ objective inputs from the arbitrary controller.
    This is not a big issue in practice, since the output of most learning algorithms are adjustable.
    \item As mentioned before introducing the formulation, we need a regularizer term $\lamalpha \barvareps \twonorm{\alpha}^2$ to keep control of the prediction error.
    Here we add $\barvareps$ by the intuition that, the regularizing effect on $\alpha$ should be smaller given a smaller noise bound.
    As an extreme case, if $\barvareps = 0$, the regularizer will vanish.
    \item Slack variable $\sigma$ is introduced to compensate for the effect of noise.
    Intuitively, we want it counteract the noise in the prediction, introduced by noisy dataset and online observation, as seen in constraint \cref{eq:robust-ddsf-direct-dynamic}.
    There is also a regularizer term $\lamsigma \twonorm{\sigma}^2$ to keep control of the slack variable.
    Here this regularizing constant is not affected by $\barvareps$, since this regularizing effect should still exist even if the noise bound is zero.
    \item It is worth-noticing that in the constraints \cref{eq:robust-ddsf-direct-dynamic,eq:robust-ddsf-direct-initial}, both the dataset and online observation are noisy, as can be seen from the tilde on $\tildey^d$ and $\tildey$.
    \item Constraint \cref{eq:robust-ddsf-direct-output} is the output constraint with tightening scheme included in \cite{berberichRobustConstraintSatisfaction2020}.
    We refer to (11) and (12) in \cite{berberichRobustConstraintSatisfaction2020} for a rigorous definition of the constraint tightening constants $a_{1,k}$ to $a_{4,k}$.
    \item Constraint \cref{eq:robust-ddsf-direct-sigma} is an auxiliary constraint to keep control of the slack variable $\sigma$.
    Intuitively, since the purpose of $\sigma$ is to compensate for the effect of noise, it should have an upper bound related to $\barvareps$.
    Also, since the noise originate from the dataset is amplified by $\alpha$, this upper bound should also be related to $\alpha$.
    It is worth-noticing that, constraint \cref{eq:robust-ddsf-direct-sigma} is not convex.
    As stated in \cite{berberichDataDrivenRobust2021}, in practice it can be omitted, since with proper choice of regularizing constants $\lamalpha$ and $\lamsigma$, it will be satisfied automatically.
    Also, from the point of the proof, it is enough to replace the constraint with a convex relaxation, as will be discussed in \cref{apps:prf-roubust-direct}.
\end{enumerate}

We can prove that the formulation \cref{eq:robust-ddsf-direct} can provide qualitative guarantees for the system.

\begin{theorem}[Qualitative Guarantees for Direct Formulation]\label{thm:robust-ddsf-direct}
    Given a controllable LTI system with noise as defined in \cref{def:lit-output-noise} along with equilibrium $\us$ and $\ys$ in the interior of $\Uset$ and $\Yset$; a dataset trajectory $\datasetSequence{\tildey}{N}$ and $\datasetSequence{u}{N}$ of the system, where $\datasetSequence{u}{N}$ is persistently exciting with order $L=2n$; and the prediction horizon $L$ in \cref{eq:robust-ddsf-direct} satisfies $L \geq 2n$; an n-step receding horizon scheme.
    And suppose the objective input satisfies that, maximum deviation from objective input is bounded, that is: $\norm{\subseq{\baru}{0}{n-1} - \subseq{\uObj}{0}{n-1}}_{\subseq{R}{0}{n-1}} \leq \barD$ for some positive constant $\barD$, for any feasible input sequence candidate $\subseq{\baru}{0}{n-1}$.

    Then, with proper choice of regularizing constants $\lamalpha$ and $\lamsigma$, there exist a constant $\varepsmax$ and a cost bound $\barV$, when the noise bound satisfies $\barvareps \leq \varepsmax$, and the optimal cost at time step $t$ is smaller than $\barV$; then recursive feasibility and closed-loop constraint satisfaction can be guaranteed after this time step for the formulation \cref{eq:robust-ddsf-direct}.
    Also the optimal cost at time step $t+n$ will also be smaller than $\barV$.
\end{theorem}

We leave the proof of \cref{thm:robust-ddsf-direct} to \cref{apps:prf-roubust-direct}.

This guarantee is called \emph{qualitative}, since it does not provide any quantitative information about the choice of regularizing constants $\lamalpha$ and $\lamsigma$, and the maximum noise bound $\varepsmax$.
We refer to \cref{apps:prf-roubust-direct} for a more detailed discussion.
In practice, $\lamalpha$ and $\lamsigma$ are two hyperparameters that can be chosen by trial and error.

Note that we need the assumption that the deviation of candidate input from objective input is bounded.
This is not a big issue in practice, since the input constraint $\Uset$ is usually known and bounded, and we can always first project the objective input to the input constraint set $\Uset$.

\subsection{Numerical Example for Direct Formulation}\label{subsec:numerical-example-direct}

In this section, we provide a numerical example to illustrate the limitations of the direct formulation \cref{eq:robust-ddsf-direct}.
To simplify the illustration, we choose an SISO system, which is the same as the one used in \cite{berberichRobustConstraintSatisfaction2020}, with transfer function:

\begin{equation*}
    G(z) = 0.01\frac{2z^2 + 6.1z + 1}{z^3 - 2.1z^2 + 1.5z - 0.3}
\end{equation*}

equipped with constraint sets $\Uset = \Yset = \left[-10, 10\right]$.
This is an LTI system with $m=p=1$ and $n=l=3$.

We choose a very small noise bound $\barvareps = 0.0001$, with regularizing constants $\barvareps\lamalpha = 500$ and $\lamsigma = 500$.
The output noise for each time step, both in dataset trajectory and online observation, is sampled uniformly random from $\left[-\barvareps, \barvareps\right]$.
The reason for this small noise bound will be clear later.

The dataset trajectory is of length $2000$, generated by sampling input uniformly random from $\Uset$ at each time step, and starting from the initial state $x_0 = \transpose{\begin{bmatrix}0 & 0 & 0\end{bmatrix}}$.
The prediction horizon is $L=20$, and we set the equilibrium point to be $\us = \ys = 0$.

By applying an open-loop objective control input of a sine wave with amplitude $15$, we observe the closed-loop input and output as shown in \cref{fig:robust-direct-conservative}.

\includefig{robust-direct-conservative.png}{1.0}{Closed-loop input and output with direct method and unsafe objective input}{fig:robust-direct-conservative}

As can be seen from \cref{fig:robust-direct-conservative}, the closed-loop input and output are far from the constraint limit.
This is due to the very conservative constraint tightening scheme used in \cite{berberichRobustConstraintSatisfaction2020}, based on a generous estimation of bound of prediction error.

Note that the constraint tightening constants $a_{1,k}$ to $a_{4,k}$ depend on the dataset trajectory, the system constants and the noise bound $\barvareps$.
To illustrate the conservative nature of this constraint tightening scheme, we list the value of $a_{4, 19}$ with different noise bound in \cref{tab:robust-direct-tightening}.

{\renewcommand{\arraystretch}{1.3}%
\begin{center}
\captionof{table}{Constraint Tightening in Direct Method \label{tab:robust-direct-tightening}}
\begin{tabular}{ r|c|c|c|c|c }
    Noise bound $\barvareps$ & 0.0001 & 0.0005 & 0.001 & 0.0015 & 0.002 \\
    \hline
    $a_{4, 19}$ & 7.411 & 418.66 & 5344.73 & 28486.42 & 99467.18 \\
\end{tabular}
\end{center}
}

As can be seen from \cref{tab:robust-direct-tightening}, the constraint tightening constant is large, and increases very quickly with the noise bound.
Remember that our output constraint is $\Yset = \left[-10, 10\right]$, so with a slightly larger noise bound, constraint \cref{eq:nominal-ddsf-output} will be trivially infeasible since $a_{4, 19} \geq 10$.
This is also the reason that we need to choose a very small noise bound $\barvareps = 0.0001$ in this example.

By applying an open-loop objective control input of a sine wave with amplitude $3$, we observe the closed-loop input and output as shown in \cref{fig:robust-direct-unwanted-interference}.

\includefig{robust-direct-unwanted-interference.png}{1.0}{Closed-loop input and output with direct method and safe objective input}{fig:robust-direct-unwanted-interference}

As can be seen from \cref{fig:robust-direct-unwanted-interference}, the closed-loop input differs slightly from the objective input, even if the objective input is safe.

This is due to the coupling of regularizing term $\lamalpha \barvareps \twonorm{\alpha}^2$ in the objective function \cref{eq:nominal-ddsf-cost} and the dynamic constraint \cref{eq:robust-ddsf-direct-dynamic}.
As can be seen from \cref{eq:robust-ddsf-direct-dynamic}, with smaller $\alpha$, the input candidate sequence $\subseq{\baru}{0}{n-1}$ will also be smaller.
The regularizing term tries to make $\alpha$ smaller and make the candidate input $\subseq{\baru}{0}{n-1}$ smaller, in order to get a smaller overall cost, even if the objective input is safe.
This is unwanted input interference.


\section{Indirect Formulation}\label{sec:indirect-formulation}

To overcome the limitations of the direct formulation, we introduce the indirect formulation of the robust data-driven safety filter, with a qualitative constraint tightening scheme.

Firstly, to deal with the conservative constraint tightening scheme, we introduce a new constraint tightening scheme, which makes the constraint tightening constants hyperparameters that we can choose.
Intuitively, since we only have qualitative guarantees for recursive feasibility and closed-loop stability, it is not necessary to have a rigorous but conservative constraint tightening scheme, as used in the constraint \cref{eq:robust-ddsf-direct-output}.

Secondly, we take advantage of the \emph{indirect method} to decompose the cost of safety filter and the dynamic constraint.

With these modifications, the direct formulation of the robust data-driven safety filter can be formulated as:

\begin{subequations}
\label{eq:robust-ddsf-indirect}
\begin{align}
    \min_{\substack{\sigma \\ \bar{u}, \bar{y}}} \quad & \norm{\subseq{\baru}{0}{n-1} - \subseq{\uObj}{0}{n-1}(t)}_{\subseq{R}{0}{n-1}}^2 + \lambda_\sigma \norm{\sigma}_2^2 \label{eq:robust-ddsf-indirect-cost} \\
    \textrm{s.t.} \quad & 
    \subseq{\bary}{0}{L-1} + \sigma 
    = \hankel{L}{\subseq{\tildey^d}{l}{N-1}} 
    \begin{bmatrix}
        \hankel{L+l}{u^d} \\
        \hankel{l}{\subseq{\tildey^d}{0}{N-L-1}} \\
    \end{bmatrix}^{\dagger}
    \begin{bmatrix}
        \subseq{\baru}{-l}{L-1} \\
        \subseq{\bary}{-l}{-1} \\
    \end{bmatrix} \label{eq:robust-ddsf-indirect-dynamic} \\
    & 
    \begin{bmatrix}
        \subseq{\baru}{-l}{-1} \\
        \subseq{\bary}{-l}{-1} \\
    \end{bmatrix} = 
    \begin{bmatrix}
        \subseq{u}{t-l}{t-1} \\
        \subseq{\tildey}{t-l}{t-1} \\
    \end{bmatrix} \label{eq:robust-ddsf-indirect-initial} \\
    & 
    \begin{bmatrix}
        \subseq{\baru}{L-n}{L-1} \\
        \subseq{\bary}{L-n}{L-1} \\
    \end{bmatrix} = 
    \begin{bmatrix}
        \vRepeatVec{\us}{n} \\
        \vRepeatVec{\ys}{n} \\
    \end{bmatrix} \label{eq:robust-ddsf-indirect-terminal} \\
    &
    \bar{u}_k \in \Uset, \quad k \in \left[0, L-1\right] \label{eq:robust-ddsf-indirect-input} \\
    &
    \norm{\bar{y}_k}_\infty + a_k \leq y_{\max}, \quad k \in \left[0, L-1\right] \label{eq:robust-ddsf-indirect-output} \\
    &
   \norm{\sigma}_\infty \leq \bar{\varepsilon}\left( \left(1+\rho_{L}^{\max}\right)\norm{\alpha}_1 + \rho_{L}^{\max} \right) \label{eq:robust-ddsf-indirect-sigma}
\end{align}
\end{subequations}

where $\alpha = \begin{bmatrix}
    \hankel{L+l}{u^d} \\
    \hankel{l}{\subseq{\tildey^d}{0}{N-L-1}} \\
\end{bmatrix}^{\dagger}
\begin{bmatrix}
    \subseq{\baru}{-l}{L-1} \\
    \subseq{\bary}{-l}{-1} \\
\end{bmatrix}$, $\rhoLmax$ is a constant of the system, as defined in (7) and discussion after (12) in \cite{berberichRobustConstraintSatisfaction2020}.
Note that $\alpha$ is not an optimization variable, it's only an auxiliary variable used to construct \cref{eq:robust-ddsf-indirect-sigma}.

The constraint tightening constants $a_k$ in \cref{eq:robust-ddsf-indirect-output} are formulated as: $a_k = \sum_{i=1}^{\ceil{\frac{k+1}{n}}} c_i$, where $\sequence{c}{k}{1}{\ceil{\frac{L}{n}}}$ is a sequence of positive constants that we can choose.
Intuitively, we are tightening the output constraint by a constant $a_k$ at each time step, and the constant $a_k$ increases by a positive constant $c_i$ every $n$ time steps.
In another word, the constraint becomes more tight every $n$ time steps.

We also need some discussion about the formulation:

\begin{enumerate}
    \item Since we are using an indirect method of making prediction, the variable $\alpha$ is not an optimization variable anymore.
    The cost \cref{eq:robust-ddsf-indirect-cost} if formulated similar to the cost in direct formulation \cref{eq:robust-ddsf-indirect-cost}, with the only difference that we do not have a regularizer term for $\alpha$.
    \item The dynamic constraint \cref{eq:robust-ddsf-indirect-dynamic} is formulated using the indirect method, as discussed in \cite{dorflerBridgingDirectIndirect2023}.
    The pseudo inverse is introduced by the closed form solution of an inner optimization problem.
    We refer to \cite{dorflerBridgingDirectIndirect2023} for more details.
    \item The initial constraint \cref{eq:robust-ddsf-indirect-initial} only uses $l$ pairs of input-output, instead of $n$ pairs as in the direct formulation.
    This is needed for the proof, and requires some knowledge about the system.
    \item As discussed before in this chapter, we design a qualitative constraint tightening scheme, which makes the constraint tightening constants $a_k$ hyperparameters that we can choose.
    \item Constraint \cref{eq:robust-ddsf-indirect-sigma} is again an auxiliary constraint to keep control of the slack variable $\sigma$.
    Since we are using an indirect method, $\sigma$ only contains $\sequence{\sigma}{k}{0}{L-1}$, corresponding to the output prediction.
    This affects our construction of candidate solution in the proof of \cref{thm:robust-ddsf-indirect}, so constraint \cref{eq:robust-ddsf-indirect-sigma} is different from \cref{eq:robust-ddsf-direct-sigma}.
    Similar discussion about this constraint can be made as in the direct formulation.
\end{enumerate}

We can prove that the formulation \cref{eq:robust-ddsf-indirect} can also provide qualitative guarantees for the system, but with slightly different assumptions from the direct formulation stated in \cref{thm:robust-ddsf-direct}.

\begin{theorem}[Qualitative Guarantees for Indirect Formulation]\label{thm:robust-ddsf-indirect}
    Given a controllable LTI system with noise as defined in \cref{def:lit-output-noise} along with equilibrium $\us$ and $\us$ in the interior of $\Uset$ and $\Yset$; a dataset trajectory $\datasetSequence{\tildey}{N}$ and $\datasetSequence{u}{N}$ of the system, where $\datasetSequence{u}{N}$ is persistently exciting with order $L=2n$; and the prediction horizon $L$ in \cref{eq:robust-ddsf-indirect} satisfies $L \geq 2n$; an n-step receding horizon scheme.
    And suppose the system constant satisfies: $n=l \times p$, and the input, output constants $\Uset$ and $\Yset$ are both finite.

    Then, with a certain choice of constraint tightening constants $\sequence{c}{k}{1}{\ceil{\frac{L}{n}}}$, there exists a constant $\varepsmax$, when the noise bound satisfies $\barvareps \leq \varepsmax$, recursive feasibility and closed-loop constraint satisfaction can be guaranteed for the formulation \cref{eq:robust-ddsf-indirect}.
\end{theorem}

We leave the proof of \cref{thm:robust-ddsf-indirect} to \cref{apps:prf-roubust-indirect}.

Note that we drop the constraint that the deviation of candidate input from objective input is bounded, which is needed in the proof of \cref{thm:robust-ddsf-direct}.

Also, we need the assumption that $n=l \times p$, which is automatically satisfied for SISO systems and for systems that we can directly observe the state.
It seems quite restrictive in theory, but in practice, even when this assumption is not satisfied, the formulation \cref{eq:robust-ddsf-indirect} can still perform well, as will be illustrated in \cref{subsec:numerical-example-indirect}.

\subsection{Numerical Example for Indirect Formulation}\label{subsec:numerical-example-indirect}

We use the same system as in \cref{subsec:numerical-example-direct}, to illustrate the effectiveness of the indirect formulation \cref{eq:robust-ddsf-indirect}.

First we use exactly the same setup as in \cref{subsec:numerical-example-direct}, and we choose the constraint tightening constants $c_i$ to be: $c_1 = 0.5$, $c_i = 0.1$ for $i \geq 2$.

The closed-loop input and output are shown in \cref{fig:robust-indirect-nonconservative} and \cref{fig:robust-indirect-no-interference}.

\includefig{robust-indirect-nonconservative.png}{1.0}{Closed-loop input and output with indirect method and unsafe objective input}{fig:robust-indirect-nonconservative}

\includefig{robust-indirect-no-interference.png}{1.0}{Closed-loop input and output with indirect method and safe objective input}{fig:robust-indirect-no-interference}

It can be seen in \cref{fig:robust-indirect-nonconservative} that, the constraint tightening scheme is not conservative anymore, and the closed-loop input and output are close to the constraint limit.
And in \cref{fig:robust-indirect-no-interference}, the closed-loop input is the same as the objective input, which is the desired behavior.

Then we change the noise bound to $\barvareps = 0.5$, in order to illustrate the effectiveness of formulation \cref{eq:robust-ddsf-indirect} under larger noise bound.
Note that the direct formulation \cref{eq:robust-ddsf-direct} will be trivially infeasible with this noise bound, since $a_{4, 19} \geq 10$.

If we use the same constraint tightening constants $c_i$ as before, the closed-loop input and output are shown in \cref{fig:robust-indirect-violation}.
The output constraint is violated, due to prediction error and not large enough constraint tightening constants.

\includefig{robust-indirect-violation.png}{1.0}{Closed-loop input and output with indirect method and too small tightening constant}{fig:robust-indirect-violation}

We can choose another set of constraint tightening constants $c_i$: $c_1 = 1.5$, $c_i = 0.2$ for $i \geq 2$.
We can see from \cref{fig:robust-indirect-noviolation} that the output constraint is satisfied, and the closed-loop input and output are close to the constraint limit.

\includefig{robust-indirect-noviolation.png}{1.0}{Closed-loop input and output with indirect method and proper tightening constant}{fig:robust-indirect-noviolation}

\newpage
Next, we change the system into a system with two outputs, to illustrate that the formulation \cref{eq:robust-ddsf-indirect} can also work, even if the assumption $n=l \times p$ is not satisfied.

To do this, we write the system into a state space model:

\begin{align*}
    A = \begin{bmatrix}
        2.1 & -1.5 & 0.3 \\
        1 & 0 & 0 \\
        0 & 1 & 0 \\
    \end{bmatrix}, \quad
    B = \begin{bmatrix}
        1 \\
        0 \\
        0 \\
    \end{bmatrix}, \quad
    C = 0.01 \begin{bmatrix}
        2 & 6.1 & 1.1 \\
        0 & 10 & 0 \\
    \end{bmatrix}, \quad
    D = \begin{bmatrix}
        0 \\
        0 \\
    \end{bmatrix}
\end{align*}

Note that in this case, $n=3$, $l=2$, $p=1$ and $m=2$.
And the first output is the same as the output of the original system, and the second output is not constrained.
We also need to modify the constraint tightening constants to be: $c_1 = 1.7$, $c_i = 0.1$ for $i \geq 2$.

We can see from \cref{fig:robust-indirect-simo} that the closed-loop input and output are close to the constraint limit, and the output constraint is satisfied.

\includefig{robust-indirect-simo.png}{1.0}{Closed-loop input and first output with indirect method, on system with two outputs}{fig:robust-indirect-simo}

The problem of how to choose the constraint tightening constants properly is left for future research.
In practice, we can always start from a conservative constraint tightening scheme, and then gradually decrease the tightening until the input interference is small enough and the constraints are still satisfied.

Also, it's worth noticing that the number of optimization variables in the indirect formulation \cref{eq:robust-ddsf-indirect} has nothing to do with the size of dataset trajectory, and is only related to the prediction horizon $L$.
This is a big advantage of the indirect formulation, since it makes the problem scalable to large dataset.
Also, due to less optimization variables, the indirect formulation is usually faster to solve than the direct formulation.
In the example above, the direct formulation takes about $1.3$ seconds to solve, while the indirect formulation only takes about $0.01$ seconds to solve.
